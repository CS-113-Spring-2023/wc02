\documentclass[a4paper]{exam}

\usepackage{geometry}
\usepackage{graphicx}
\usepackage{hyperref}
\usepackage{titling}

\printanswers

\title{Weekly Challenge 02: Formal Reasoning\\CS/MATH 113 Discrete Mathematics}
\author{$\langle team-name \rangle$}  % <== for grading, replace with your team name, e.g. q1-team-420
\date{Habib University | Spring 2023}

\qformat{{\large\bf \thequestion. \thequestiontitle}\hfill}
\boxedpoints

\begin{document}
\maketitle

\begin{questions}
  
\titledquestion{Logical Hikers}
Imagine a group of three logical hikers standing in a circle at the base of a mountain. The guide enters the group with a bag of compasses and informs the 
hikers that each compass is either working or broken. Then walking behind the hikers, he places a compass in each hiker's pocket. No hiker can see the 
condition of their compass, but they can see every other hiker's compass if they take it out to check. We suppose that all the compasses are working. 
Now the guide states that the hiker who can correctly identify the condition of their compass will lead the group, and any hiker who is incorrect will 
have to carry the extra load to discourage guessing. The guide then progresses repeatedly around the circle, asking each hiker in turn whether they would 
like to state the condition of their compass.


\begin{parts}
  \part Will any hiker ever be able to state the condition of their compass ? Justify your answer.
  \part Now suppose the guide announces to the group that at least one of the compasses is working and then once again progresses repeatedly around the circle asking each hiker in turn whether 
  they would now like to state the condition of their compass. Will any hiker ever be able state the condition of their compass ? Justify your answer. 
  
\end{parts}

\begin{solution}
    % Enter your solution here.
  \end{solution}
\end{questions}


\end{document}

%%% Local Variables:
%%% mode: latex
%%% TeX-master: t
%%% End:
